\documentclass[margin,line]{res}


\oddsidemargin -.5in
\evensidemargin -.5in
\textwidth=6.0in
\itemsep=0in
\parsep=0in
% if using pdflatex:
%\setlength{\pdfpagewidth}{\paperwidth}
%\setlength{\pdfpageheight}{\paperheight} 

\newenvironment{list1}{
  \begin{list}{\ding{113}}{%
      \setlength{\itemsep}{0in}
      \setlength{\parsep}{0in} \setlength{\parskip}{0in}
      \setlength{\topsep}{0in} \setlength{\partopsep}{0in} 
      \setlength{\leftmargin}{0.17in}}}{\end{list}}
\newenvironment{list2}{
  \begin{list}{$\bullet$}{%
      \setlength{\itemsep}{0in}
      \setlength{\parsep}{0in} \setlength{\parskip}{0in}
      \setlength{\topsep}{0in} \setlength{\partopsep}{0in} 
      \setlength{\leftmargin}{0.2in}}}{\end{list}}


\begin{document}

\name{Ragesh Kumar Ramachandran \vspace*{.1in}}

\begin{resume}
\section{\sc Contact Information}
\vspace{.05in}
\begin{tabular}{@{}p{3.5in}p{4in}}
Robotics Embedded Systems Lab (RESL)             &  \\            
Department of Computer Science   & \\  %{\it Fax:}       
University of Southern California & \\       
Los Angeles, CA  90007 USA  &  \\     %{\it WWW:} www.stat.cmu.edu/\verb+~+paciorek
{\it Voice:}  (480) 522-9070 & \\
{\it E-mail:}  rageshku@usc.edu & \\
{\it Website:}  https://ragesh88.github.io/ & \\
\end{tabular}


\section{\sc Research Interests}
Swarm robotics, resilience in multi-robot teams, optimal and nonlinear control theory, network and graph  theory,  applied topology and differential geometry, and inverse problems.

\section{\sc Education}
{\bf Arizona State University}, Tempe, Arizona USA\\
%{\em Department of Statistics} 
\vspace*{-.1in}
\begin{list1}
\item[] Ph.D. , Mechanical Engineering, \textit{GPA } : 4.00/4.00, \textbf{August 2012} - \textbf{August 2018}
\begin{list2}
\vspace*{.05in}
\item Dissertation Topic:  ``Exploration, Mapping and Scalar Field Estimation using a Swarm of Resource-Constrained Robots.'' 

\item Advisor:  Spring M. Berman
\end{list2}
\vspace*{.05in}
%\item[] M.S., Statistics,  May 2000
\end{list1}

%{\bf Duke University}, Durham, North Carolina USA\\
%%{\em Department of Mathematics and Statistics} 
%\vspace*{-.1in}
%\begin{list1}
%\item[] M.S., Botany (Ecology),  May, 1998
%\end{list1}

{\bf National Institute of Technology Calicut}, Calicut, Kerala India\\
%{\em Department of Mathematics and Statistics} 
\vspace*{-.1in}
\begin{list1}
\item[] Bachelor of Technology, Civil Engineering, \textit{GPA} : 7.11/10.00, May, 2011
\end{list1}


%\section{\sc Honors and Awards} 
%National Science Foundation Graduate Research Fellowship, 1996

%\vspace*{-2.5mm}
%NSF Vertical Integration of Research and Education in Statistics and
%Mathematical Sciences\\ (VIGRE) teaching fellowship.
%
%\vspace*{-2.5mm}
%Carleton College: graduated Magna Cum Laude, Honors in Biology, Phi Beta Kappa, 1993

\section{\sc Academic Experience}
{\bf University Southern California}, Los Angeles, California USA

\vspace{-.3cm}
{\em Postdoctoral Scholar - Research Associate} \hfill {\bf August, 2018 - present}\\
Advisor:  Gaurav Sukhatme

{\bf Massachusetts Institute of Technology}, Boston, Massachusetts USA

\vspace{-.3cm}
{\em Visiting Research Scholar} \hfill {\bf August, 2019}\\
Advisor:  Sertac Karaman


{\bf Arizona State University}, Tempe, Arizona USA

\vspace{-.3cm}
{\em Graduate Student} \hfill {\bf August, 2012 - August, 2018}\\
Includes current Ph.D.~research, Ph.D.~and Masters level coursework and
research/consulting projects.

%\vspace{-.1cm}
%{\em Instructor} \hfill {\bf May - June, 2002}\\
%Co-taught graduate level course for the Master of Science in
%Computational Finance program.  Shared responsibility for lectures, exams,
%homework assignments, and  grades.  
%\vspace*{.05in}  
%\begin{list2}
%\item 46-731 Probability and Statistics, Summer 2002.
%\end{list2}


%\vspace{-.1cm}
%{\em NSF VIGRE Teaching Fellow} \hfill {\bf January - May, 2001}\\
%Head teaching assistant.   
%Duties included  shared administrative responsibilities with faculty
%instructor, fielding of all student inquiries, and oversight of
%graduate student teaching assistants and graders.
%\vspace*{.05in}  
%\begin{list2}
%\item 36-217 Probability Theory and Random Processes, Spring 2001.
%\end{list2}

%\vspace{-.1cm}


\section{\sc Teaching Experience}
{\bf University Southern California}, Los Angeles, California USA

\vspace{-.3cm}
{\em Introduced a course titled: ``Applied Mathematics in Robotics''} \hfill {\bf Summer 2019}\\

{\bf Arizona State University} \\
{\em Teaching Assistant}\hfill {\bf January, 2013  - May, 2014}\\
Duties at various times have included 
office hours and leading weekly lab exercises.
\begin{list2}
	\item MAE 322 Structural Mechanics, Spring 2013.
	\item MAE 419 Experimental Mechanical Engineering, Fall 2013.
	\item MAE 318 System Dynamics and Control, Spring 2014.
\end{list2}

\section{\sc Journal Publications}
\begin{enumerate}
	\item \textbf{Ragesh K. Ramachandran}, Zahi Kakish and Spring Berman. Information correlated L\'evy walk exploration and distributed mapping using a swarm of robots. \textit{IEEE Transactions on Robotics} (T-RO), 2020.	
	
	\item \textbf{Ragesh K. Ramachandran}, Sean. Wilson, and Spring. Berman. A probabilistic approach to automated construction of topological maps using a stochastic robotic swarm. \textit{IEEE Robotics and	Automation Letters}, 2(2):616–623, April 2017.
	
	\item Thomas G. Sugar, Andrew Bates, Matthew Holgate, Jason Kerestes, Marc. Mignolet, Philip. New, \textbf{Ragesh K. Ramachandran}, Sangram. Redkar, Chase. Wheeler, (2015). Limit cycles to enhance human performance based on phase oscillators. \textit{Journal of Mechanisms and Robotics}, 7, 011001.
	
\end{enumerate}


%%%%%%%%%%%%%%%%%%%%%%%%%%Papers in preparation%%%%%%%%%%%%%%%%%%%%%%%%%%%%%%%%%%%%%

%\section{\sc Papers in preparation}
%
%Ventura, V., C.J. Paciorek, and J.S. Risbey.  Controlling the proportion of falsely-rejected hypotheses when conducting multiple tests with geophysical data.
%
%Ickes, K., C.J. Paciorek, and S. Thomas.  Effects of wild pigs on
%forest demographic processes in Malaysia.



\section{\sc Peer Reviewed Conference Publications}
\begin{enumerate}
	\item \textbf{Ragesh K. Ramachandran}, Lifeng Zhou, James A. Preiss and Gaurav S. Sukhatme. Resilient  Coverage:  Exploring  the  Local-to-Global  Trade-off. Accepted to \textit{ IEEE/RSJ International Conference on Intelligent Robots and Systems (IROS)} 2020.	
	
	\item Renato Fernando dos Santos, \textbf{Ragesh K. Ramachandran}, Marcos A. M. Vieira and Gaurav S. Sukhatme. Pac-Man is Overkill. Accepted to \textit{ IEEE/RSJ International Conference on Intelligent Robots and Systems (IROS)}, Las Vegas, USA, 2020.	
	
	\item \textbf{Ragesh K. Ramachandran}, Nicole Fronda and Gaurav S. Sukhatme. Resilience in multi-robot target tracking through reconfiguration. In Proceedings of the \textit{ IEEE International Conference on Robotics and Automation (ICRA)}, Paris, France, May  2020.
	
	\item Eric Heiden and Ziang Liu and \textbf{Ragesh K. Ramachandran} and Gaurav S. Sukhatme. Physics-based  Simulation  of  Continuous-Wave  LIDAR
	for  Localization,  Calibration  and  Tracking. Submitted to \textit{ IEEE International Conference on Robotics and Automation (ICRA)}, Paris, France, May   2019.
	
	\item \textbf{Ragesh K. Ramachandran} and Spring Berman. Automated Construction of Metric Maps using a Stochastic Robotic Swarm Leveraging Received Signal Strength. Proceedings of the \textit{International Symposium on Swarm Behavior and Bio-Inspired Robotics (SWARM)} 2019, Okinawa, Japan, November 20–22, 2019.
	
	\item \textbf{Ragesh K. Ramachandran}, James A. Preiss and Gaurav S. Sukhatme. Resilience by Reconfiguration: Exploiting Heterogeneity in Robot Teams. In Proceedings of the \textit{ IEEE/RSJ International Conference on Intelligent Robots and Systems (IROS)}, Macau, China, November 4–8, 2019.
	
	\item \textbf{Ragesh K. Ramachandran} and Spring Berman. The effect of communication topology on scalar field estimation by large networks with partially accessible measurements. In Proceedings of the \textit{ American Control Conference (ACC)}, Seattle, WA, USA, May 24–26, 2017.
	
	\item \textbf{Ragesh K. Ramachandran},  Sean Wilson and Spring Berman. A probabilistic topological approach to feature identification using a stochastic robotic swarm. In the Proceedings of \textit{International Symposium on Distributed Autonomous Robotic Systems (DARS)}, London, UK, November 7-9, 2016. (Accepted for oral presentation - 25\% acceptance rate)
	
	\item \textbf{Ragesh K. Ramachandran}, Karthik Elamvazhuthi, and Spring Berman. An optimal control approach to mapping GPS-denied environments using a stochastic robotic swarm. In \textit{International Symposium on Robotics Research (ISRR)}, 2015.
	
	\item Jason Kerestes, Thomas G Sugar, Thierry Flaven, Matthew Holgate, and \textbf{Ragesh K Ramachandran}. A method to add energy to running gait: Pogosuit. In ASME 2014 International Design Engineering Technical Conferences and Computers and Information in Engineering Conference, pages V05AT08A005-V05AT08A005. \textit{American Society of Mechanical Engineers}, 2014.
	
	\item \textbf{Ragesh K. Ramachandran}, Vivek M Elayidom, A P Sudheer. (2009). Target Location algorithm for Automation Assistance in Weld Industries. The proceedings of the \textit{International Conference on Simulation Modeling and Analysis (COSMA}) , NIT Calicut, December 2009: 194 - 198.
	
\end{enumerate}

%\section{\sc Journal in Preparation}
%\begin{enumerate}
%	\item \textbf{Ragesh K. Ramachandran} and Spring Berman. "Automated Construction of Metric Maps using a Stochastic Robotic Swarm Leveraging Received Signal Strength". In preparation for IEEE Robotics and Automation Letters (RA-L), 2018.
%\end{enumerate}

\section{\sc Submitted Journal Papers}
\begin{enumerate}	
	\item \textbf{Ragesh K. Ramachandran}, Nicole Fronda and Gaurav S. Sukhatme. Resilience in multi-robot multi-target tracking with unknown number of targets through reconfiguration. Submitted to \textit{IEEE Transactions on Control of Network Systems} (TCNS), 2020.
	
	
\end{enumerate}

%\section{\sc Submitted Conference Papers}
%\begin{enumerate}	
%	
%	
%			
%	
%	
%\end{enumerate}


%\vspace*{-.25in}  
%\begin{verbatim}http://www.nist.gov/speech/publications/tw00/html/abstract.htm#cp1-50\end{verbatim}

\section{\sc Refereed Abstracts}
\begin{enumerate}
	\item \textbf{Ragesh Kumar Ramachandran}, and James A. Preiss and Gaurav S. Sukhatme. ``Resilience by Reconfiguration: Exploiting Heterogeneity in Robot Teams''. Workshop on Resilient Robot Teams:	Composing, Acting, and Learning, ICRA 2019: International Conference on Robotics and Automation, Montreal, Canada, 2019.
	
	\item \textbf{Ragesh Kumar Ramachandran}, and Spring Berman. ``Post Processing of Occupancy Grid Maps using Persistent Homology''. Workshop on Emerging Topological Techniques in Robotics, ICRA 2019: International Conference on Robotics and Automation, Montreal, Canada, 2019.
	
	\item \textbf{Ragesh Kumar Ramachandran}, and Spring Berman. ``Topological Mapping Using a Heterogeneous Robotic Swarm''. Workshop on Emerging Topological Techniques in Robotics, ICRA 2016: International Conference on Robotics and Automation, Stockholm, Sweden, 2016.  
\end{enumerate}
\section{\sc Invited talks}
\begin{list2}
	\item ``Estimation and Mapping using a Swarm of Resource-Constrained Robots" School of Electrical and Electronic Engineering, Nanyang Technological University, Singapore, January 16, 2018.
	\item ``Topological Mapping using a Stochastic Robotic Swarm" Mechanical \& Aerospace Engineering Seminar, Arizona State University, Tempe, Arizona USA, April 7, 2017.
\end{list2}

\section{\sc Workshops organized}
\begin{list2}
	\item ``Heterogeneous Multi-Robot Task Allocation and Coordination" Harish Ravichandar, \textbf{Ragesh Kumar Ramachandran}, Sonia Chernova, Seth Hutchinson, Gaurav Sukhatme, and Vijay Kumar. Robotics: Science and Systems, 2020.
\end{list2}

%\section{\sc Conferences Attended}
%\begin{list2}
%	\item American Control Conference (ACC), May 2017, Seattle, WA, USA.
%	\item IEEE Conference on Decision and Control (CDC), December 2016, Las Vegas, NV, USA.
%	\item International Symposium on Distributed Autonomous Robotic Systems (DARS), November 2016, London, UK.
%	\item IEEE International Conference on Robotics and Automation (ICRA), May 2016, Stockholm, Sweden. 
%	\item International Symposium on Robotics Research (ISRR), September 2015, Sestri Levante (Genova), Italy.
%\end{list2}


%%%%%%%%%%%%%%%% Professional Experience %%%%%%%%%%%%%%%%%%%%%%%%%%%%%%%%%

%\section{\sc Professional Experience}
%{\bf Bureau of Transportation Statistics, U.S. Department of
%  Transportation}, Washington, District of Columbia USA
%
%\vspace{-.3cm}
%{\em Summer researcher} \hfill {\bf May, 2000 - August, 2000}\\
%Carried out several consulting projects, including modelling of
%injuries to cadavers in crash test experiments, analysis of airline
%delay data, and advice on analysis of airline economics data.
%
%{\bf Abt Associates}, Bethesda, Maryland USA
%
%\vspace{-.3cm}
%{\em Associate Programmer Analyst and Research Assistant} \hfill {\bf
%  October, 1994 - August, 1996}\\
%Researcher and computer model developer for U.S. EPA Regulatory Impact
%Analysis of Section 403 Lead Paint Hazard Rule.  Other projects
%included database analysis, literature reviews, and cost-benefit analysis.

\section{\sc Professional Service}
\begin{list2}
	\item \textbf{Mentoring :} Mentored a graduate student (Nicole Fronda) in implement a resilient strategy for distributed target tracking. Nicole Fronda received the \textbf{best student researcher award} in computer science.\\
	Mentored a graduate student(Vaibhav Deshmukh) in implementing a decentralized Markov chain based strategy on our robotic platform Pheeno using ROS. 
	\item \textbf{Journal Review :} Computers \& Graphics2020, Robotica 2019, IEEE Robotics and Automation Letters 2019, Swarm Intelligence 2019, IEEE Control Systems Letters 2018, IEEE Transactions on Automation Science and Engineering 2018, Autonomous Robots 2017.
	\item \textbf{Conference Review :} IEEE International Conference on Robotics and Automation (ICRA) 2020, Robotics: Science and Systems(RSS) 2020, Robotics: Science and Systems(RSS) 2019, International Conference on Intelligent Robots and Systems (IROS) 2018, IEEE International Conference on Robotics and Automation (ICRA) 2016, International Symposium on Distributed Autonomous Robotic Systems (DARS) 2016 and International Conference on Intelligent Robots and Systems (IROS) 2015. 
\end{list2}

%%%%%%%%%%%%%%%%%%%%%Peer-reviewed conference presentations%%%%%%%%%

\section{\sc Peer-reviewed conference presentations}
(\textit{P}) = Presenter of a talk or poster
\begin{list2}
	\item \textbf{Ragesh K. Ramachandran}\textit{P} and Spring Berman. Automated Construction of Metric Maps using a Stochastic Robotic Swarm Leveraging Received Signal Strength. Proceedings of the \textit{International Symposium on Swarm Behavior and Bio-Inspired Robotics (SWARM)} 2019, Okinawa, Japan, November 20–22, 2019.
	
	\item \textbf{Ragesh K. Ramachandran}\textit{P}, James A. Preiss and Gaurav S. Sukhatme. Resilience by Reconfiguration: Exploiting Heterogeneity in Robot Teams. In Proceedings of the \textit{ IEEE/RSJ International Conference on Intelligent Robots and Systems (IROS)}, Macau, China, November 4–8, 2019.
	
	\item \textbf{Ragesh K. Ramachandran}\textit{P} and Spring Berman. The effect of communication topology on scalar field estimation by large networks with partially accessible measurements. In Proceedings of the 2017\textit{ American Control Conference (ACC)}, Seattle, WA, USA, May 24–26, 2017. Oral presentation.
	\item \textbf{Ragesh K. Ramachandran}\textit{P}, Sean Wilson, and Spring Berman. A probabilistic topological approach to feature identification using a stochastic robotic swarm. In To appear in the \textit{International Symposium on Distributed Autonomous Robotic Systems (DARS)}, 2016. Oral presentation.
	\item \textbf{Ragesh Kumar Ramachandran}\textit{P}, and Spring Berman. ''Topological Mapping Using a Heterogeneous Robotic Swarm''. Workshop on Emerging Topological Techniques in Robotics, ICRA 2016: International Conference on Robotics and Automation, Stockholm, Sweden, 2016.  Poster presentation.
	\item \textbf{Ragesh K. Ramachandran}\textit{P}, Karthik Elamvazhuthi, and Spring Berman. An optimal control approach to mapping GPS-denied environments using a stochastic robotic swarm. In \textit{International Symposium on Robotics Research (ISRR)}, 2015. Oral presentation.
\end{list2}


%%%%%%%%%%%%%%%%%%%%% SOFTWARE SKILLS %%%%%%%%%%%%%%%%%%%%%%%%%%%%%%%%%%

\section{\sc Computer Skills} 
\begin{list2}
\item \textbf{Languages/Software:} Python, C/C++, Matlab, Java, Basic, Visual Basic, Unix shell scripts, ROS, \LaTeX, HTML, CSS, Javascript
%\item \textbf{Operating Systems:}  Unix/Linux, MacOS, Windows.
\item \textbf{Version control:} git, mercurial, svn
\end{list2}

%\section{\sc References}
%
%\textbf{Gaurav Sukhatme}\\
%Executive Vice Dean of Engineering \\
%Viterbi School of Engineering\\
%University of Southern California\\
%E-mail: gaurav@usc.edu\\
%Phone: (213) 740-4498\\
%
%\textbf{Spring Berman}\\
%Assistant Professor\\
%School Engineering Matter Transport Energy\\
%Arizona State University\\
%E-mail: spring.berman@asu.edu\\
%Phone:  (480) 965-4431\\
%
%\textbf{Marc Mignolet}\\
%Professor\\
%Graduate Program Chair Mechanical Engineering\\
%School Engineering Matter Transport Energy\\
%Arizona State University\\
%E-mail: marc03@asu.edu\\
%Phone: (480) 965-1484\\
%
%%\textbf{Douglas Cochran}\\
%%Associate Professor\\
%%Electrical, Computer and Energy Engineering\\
%%Arizona State University\\
%%E-mail: cochran@asu.edu\\
%%Phone: (480) 965-7409\\
%
%%\textbf{Joseph K. Davidson}\\
%%Emeritus Professor\\
%%School Engineering Matter Transport Energy\\
%%Arizona State University\\
%%E-mail: J.DAVIDSON@asu.edu\\
%%Phone: (480) 965-3824\\
%%
%\textbf{Michael Robinson}\\
%Assistant Professor\\
%Department of Mathematics and Statistics\\
%American University\\
%E-mail: michaelr@american.edu\\
%Phone: (202)885-3681\\

\end{resume}
\end{document}




